\documentclass[runningheads]{llncs}
\usepackage{hyperref}

\usepackage{amsmath, amssymb}
% For moving the title up
\usepackage{titling}
\usepackage{mystyles}
\usepackage{mymacros}

\usepackage{float}
\usepackage{caption}
\usepackage{chngcntr}
\usepackage{tikz}
\usepackage{stmaryrd}

\captionsetup{labelfont=bf, justification=raggedright, singlelinecheck=false}
\counterwithin{figure}{section}
\DeclareCaptionType[fileext=los,placement=H]{protocol}
\counterwithin{protocol}{section}
\DeclareCaptionType[fileext=los,placement=H]{functionality}
\counterwithin{functionality}{section}

\newenvironment{afigure}[2]
    {
    \begin{figure}[H]
    \begin{mdframed}[align=left]
    \caption{\label{#1} \textbf{#2}}
    }
    {
    \end{mdframed}
    \end{figure}
    }

\newenvironment{aprotocol}[2]
    {
    \begin{protocol}
    \begin{mdframed}[align=left]
    \caption{\label{#1} \textbf{#2}}
    }
    {
    \end{mdframed}
    \end{protocol}
    }

\newenvironment{afunctionality}[2]
    {
    \begin{functionality}
    \begin{mdframed}[align=left]
    \caption{\label{#1} \textbf{#2}}
    }
    {
    \end{mdframed}
    \end{functionality}
    }


\date{\today}
\title{On Security Against Time Traveling Adversaries}
\author{Lúcás Críostóir Meier\\\texttt{lucas@cronokirby.com}}

\begin{document}

\maketitle

\begin{abstract}
    \noindent In this work, we investigate the notion of
    time travel, formally defining a model for adversaries
    equipped with a time machine, and the subsequent consequences
    on common cryptographic schemes.
\end{abstract}

\section{Introduction}

\cite{maurer_unifying_2009}

\section{Defining Abstract Games}

\begin{game}
\captionsetup{justification=centering}
$$
\boxed{
\begin{aligned}
&\colorbox{pink}{\large $G_b(\text{init}, \text{next})$}\cr
\cr
&s \gets \text{init}()\cr
&\cr
&\underline{\mathcal{O}(x):}\cr
&\ s, y \gets \text{next}(b, s, x)\cr
&\ \texttt{return } y
\end{aligned}
}
$$
\caption{$G_b(\text{init}, \text{next})$}
\end{game}

\section{Models of Time Travel}

\subsection{Rewinding Models}

\subsection{Forking Models}

\subsection{Summary}

\section{On Depth and Position Restrictions}

\section{Effects of Time Travel on Common Schemes}

\subsection{Stateless Schemes Remain Secure}

\subsection{On Encryption}

\subsection{On Signatures}

\section{Further Work}

\section{Conclusion}

\bibliographystyle{alpha}
\small \bibliography{bib}
\end{document}
