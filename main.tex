\documentclass[runningheads]{llncs}
\usepackage{hyperref}

\usepackage{amsmath, amssymb}
% For moving the title up
\usepackage{titling}
\usepackage{mystyles}
\usepackage{mymacros}

\usepackage{float}
\usepackage{caption}
\usepackage{chngcntr}
\usepackage{tikz}
\usepackage{stmaryrd}

\captionsetup{labelfont=bf, justification=raggedright, singlelinecheck=false}
\counterwithin{figure}{section}
\DeclareCaptionType[fileext=los,placement=H]{protocol}
\counterwithin{protocol}{section}
\DeclareCaptionType[fileext=los,placement=H]{functionality}
\counterwithin{functionality}{section}

\newenvironment{afigure}[2]
    {
    \begin{figure}[H]
    \begin{mdframed}[align=left]
    \caption{\label{#1} \textbf{#2}}
    }
    {
    \end{mdframed}
    \end{figure}
    }

\newenvironment{aprotocol}[2]
    {
    \begin{protocol}
    \begin{mdframed}[align=left]
    \caption{\label{#1} \textbf{#2}}
    }
    {
    \end{mdframed}
    \end{protocol}
    }

\newenvironment{afunctionality}[2]
    {
    \begin{functionality}
    \begin{mdframed}[align=left]
    \caption{\label{#1} \textbf{#2}}
    }
    {
    \end{mdframed}
    \end{functionality}
    }


\date{\today}
\title{MPC for Group Reconstruction Circuits}
\author{Lúcás Críostóir Meier}

\begin{document}

\maketitle

\begin{abstract}
    In this paper, we present a thing.
\end{abstract}

\section{Introduction}

\todo{Write the introduction}

\section{Background}

Throughout this paper, we let $\G$ denote a group of prime order $q$,
with generators $G$ and $H$. Let $\Fq$ denote the scalar field associated
with this group, and let $\Zq$ denote the additive group of elements
in this field.

We make heavy use of group homomorphisms throughout this paper.
We let
$$
\varphi(P_1, \ldots, P_m) : \mathbb{A} \to \mathbb{B}
$$
denote a homomorphism from $\mathbb{A}$ to $\mathbb{B}$, parameterized
by some public values $P_1, \ldots, P_m$. Commonly $\mathbb{A}$
will be a product of several groups $\mathbb{G}_1, \ldots, \mathbb{G}_n$,
in which case we'd write:
$$
\varphi(P_1, \ldots, P_m)(x_1, \ldots, x_n)
$$
to denote the application of $\varphi$ to an element $(x_1, \ldots, x_n)$
of the product group. We also often leave the public values $P_i$ implicit.

\subsection{Pedersen Commitments}

\subsection{Sigma Protocols}

\subsection{Maurer's \texorpdfstring{$\varphi$}{varphi}-Proof}

In \cite{maurer_unifying_2009}, Maurer generalized Schnorr's sigma 
protocol for knowledge of the discrete logarithm \todo{cite} to a much larger class
of relations. In particular, Maurer provided a sigma protocol for
proving knowledge of the pre-image of a group homomorphism $\varphi$.
We denote this protocol as a ``$\varphi$-proof'', and recapitulate the scheme
here.

\subsection{UC Security and the Hybrid Model}

\subsection{Ideal Functionalities for Sigma Protocols}

\subsection{Broadcast Functionalities}

\section{Group Reconstruction Circuits}

\subsection{Formal Definition}

\subsection{Normalized Form}

\section{MPC Protocol for GRCs}

\section{Security Analysis}

\section{Applications}

\section{Limitations and Further Work}

\section{Conclusion}

\bibliographystyle{alpha}
\small \bibliography{bib}
\end{document}
