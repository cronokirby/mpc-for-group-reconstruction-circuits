\documentclass[12pt]{article}
\usepackage[a4paper, margin=1.5in]{geometry}
\usepackage{hyperref}
\usepackage{fancyhdr}
\usepackage[parfill]{parskip}

\usepackage[T1]{fontenc}
\usepackage{newtxtext}
\usepackage{amsmath,amssymb}
\let\openbox\relax
\usepackage{amsthm}

% For moving the title up
\usepackage{titling}
\setlength{\droptitle}{-4em}
\usepackage{mystyles}
\usepackage{mymacros}

\usepackage{float}
\usepackage{caption}
\usepackage{chngcntr}
\usepackage{tikz}
\usepackage{stmaryrd}

\captionsetup{labelfont=bf, justification=raggedright, singlelinecheck=false}
\counterwithin{figure}{section}
\DeclareCaptionType[fileext=los,placement=H]{protocol}
\counterwithin{protocol}{section}
\DeclareCaptionType[fileext=los,placement=H]{functionality}
\counterwithin{functionality}{section}

\newenvironment{afigure}[2]
    {
    \begin{figure}[H]
    \begin{mdframed}[align=left]
    \caption{\label{#1} \textbf{#2}}
    }
    {
    \end{mdframed}
    \end{figure}
    }

\newenvironment{aprotocol}[2]
    {
    \begin{protocol}
    \begin{mdframed}[align=left]
    \caption{\label{#1} \textbf{#2}}
    }
    {
    \end{mdframed}
    \end{protocol}
    }

\newenvironment{afunctionality}[2]
    {
    \begin{functionality}
    \begin{mdframed}[align=left]
    \caption{\label{#1} \textbf{#2}}
    }
    {
    \end{mdframed}
    \end{functionality}
    }


\date{\today}
\title{MPC for Group Reconstruction Circuits}
\author{Lúcás Críostóir Meier}

\begin{document}

\maketitle

\begin{abstract}
    In this paper, we present a thing.
\end{abstract}

\section{Introduction}

\todo{Write the introduction}

\section{Background}

Throughout this paper, we let $\G$ denote a group of prime order $q$,
with generators $G$ and $H$. Let $\Fq$ denote the scalar field associated
with this group, and let $\Zq$ denote the additive group of elements
in this field.

We make heavy use of group homomorphisms throughout this paper.
We let
$$
\varphi(P_1, \ldots, P_m) : \mathbb{A} \to \mathbb{B}
$$
denote a homomorphism from $\mathbb{A}$ to $\mathbb{B}$, parameterized
by some public values $P_1, \ldots, P_m$. Commonly $\mathbb{A}$
will be a product of several groups $\mathbb{G}_1, \ldots, \mathbb{G}_n$,
in which case we'd write:
$$
\varphi(P_1, \ldots, P_m)(x_1, \ldots, x_n)
$$
to denote the application of $\varphi$ to an element $(x_1, \ldots, x_n)$
of the product group. We also often leave the public values $P_i$ implicit.

\subsection{Pedersen Commitments}

\subsection{Sigma Protocols}

\subsection{Maurer's \texorpdfstring{$\varphi$}{varphi}-Proof}

In \cite{maurer_unifying_2009}, Maurer generalized Schnorr's sigma 
protocol for knowledge of the discrete logarithm \cite{schnorr_efficient_1990} to a much larger class
of relations. In particular, Maurer provided a sigma protocol for
proving knowledge of the pre-image of a group homomorphism $\varphi$.
We denote this protocol as a ``$\varphi$-proof'', and recapitulate the scheme
here.

Given a homomorphism $\varphi : \mathbb{A} \to \mathbb{B}$, and a public value
$X \in \mathbb{B}$, the prover wants to demonstrate knowledge of a private
value $x \in \mathbb{A}$ such that $\varphi(x) = X$. The prover
does this by means of Protocol \ref{prot:phiproof}:

\begin{aprotocol}{prot:phiproof}{$\varphi$-Proof}
\[
\begin{aligned}
    &\textbf{Prover}&&\textbf{Verifier}\\
    &\text{knows } x \in \mathbb{A}&&\text{public } X \in \mathbb{B}\\
    \\
    &k \xleftarrow{R} \mathbb{A}&\\
    &K \gets \varphi(k)&\\
    &&\quad\overset{K}{\longrightarrow}\quad\\
    &&&c \xleftarrow{R} \mathbb{Z}/(p)\\
    &&\quad\overset{c}{\longleftarrow}\quad\\
    &s \gets k + c \cdot x\\
    &&\quad\overset{s}{\longrightarrow}\quad\\
    &&& \varphi(s) \overset{?}{=} K + c \cdot X\\
\end{aligned}
\]
\end{aprotocol}

Here, $p$ is chosen such that $\forall B \in \mathbb{B}.\ p \cdot B = 0$.
In practice, we'll set $p = q$, which will work perfectly for the groups
we use, which are all products of $\G$ or $\Zq$.

\begin{claim}
    Protocol \ref{prot:phiproof} is a valid sigma protocol.
\end{claim}

Completeness follows directly from the fact that $\varphi$ is a homomorphism.

For the HVZK property, the simulator $\mathcal{S}(X, c)$ works by generating
a random $s \xleftarrow{R} \mathbb{A}$, and then setting $K := \varphi(S) - c \cdot X$.

Finally, we prove $2$-extractability. Given two verifying transcripts
$(K, c, s)$ and $(K, c', s')$ sharing the first message, we extract
a value $\hat{x}$ satisfying $\varphi(\hat{x}) = X$ as follows:

$$
\begin{aligned}
\varphi(s) - c \cdot X &= K = \varphi(s') - c' \cdot X\\
\varphi(s) - \varphi(s') &= c \cdot X - c' \cdot X\\
\frac{1}{c - c'} \cdot \varphi(s - s') &= X\\
\varphi \left(\frac{s - s'}{c - c'}\right) &= X
\end{aligned}
$$

Thus, defining $\hat{x} := (s - s') / (c - c')$, we successfully extract
a valid pre-image.

We conclude that the protocol is a valid sigma protocol.

$\blacksquare$

Maurer's protocol can also work even in the case where the order of
the groups are not known, but this makes the challenge generation
a bit more complicated, and we don't need this functionality in
this work.

\subsection{UC Security and the Hybrid Model}

\subsection{Ideal Functionalities for Sigma Protocols}

\begin{afunctionality}{fun:zk}{Zero-Knowledge Functionality $\mathcal{F}(\texttt{ZK}, \varphi)$}
A functionality $\mathcal{F}$ for parties $P_1, \ldots, P_n$.\\
\\
On input $(\texttt{prove}, \sid, x)$ from $P_i$:\\
$\mathcal{F}$ checks that $\sid$ has not been used by $P_i$ before.\\
$\mathcal{F}$ generates a new token $\pi$, and sets $x_\pi \gets x$.\\
$\mathcal{F}$ replies with $(\texttt{proof}, \pi)$.\\
\\
On input $(\texttt{verify}, X, \pi)$:\\
$\mathcal{F}$ replies with $(\texttt{verify-result}, \varphi(x_\pi) \stackrel{?}{=} X)$.\\


\end{afunctionality}

\subsection{Broadcast Functionalities}

\begin{afunctionality}{fun:broadcast}{Authenticated Broadcast Functionality $\mathcal{C}$}
A functionality $\mathcal{C}$ for parties $P_1, \ldots, P_n$.\\
\\
On receiving $(\texttt{broadcast-in}, \sid, m)$ from $P_i$:\\
$\mathcal{C}$ checks that $\sid$ has not been used by $P_i$ before.\\
$\mathcal{C}$ sends $(\texttt{broadcast-out}, \pid_i, \sid, m)$ to every party $P_j$.
\end{afunctionality}

\section{Group Reconstruction Circuits}

\subsection{Formal Definition}

\subsection{Normalized Form}

\section{MPC Protocol for GRCs}

\subsection{Ideal Functionality}

\subsection{Protocol}

\subsection{Security Analysis}

\subsection{Practical Considerations}

\section{Applications}

\section{Limitations and Further Work}

\section{Conclusion}

\bibliographystyle{alpha}
\small \bibliography{bib}
\end{document}
